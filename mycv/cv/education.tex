%!TEX root = ../cv.tex

%-------------------------------------------------------------------------------
%	SECTION TITLE
%-------------------------------------------------------------------------------
\cvsection{Education}


%-------------------------------------------------------------------------------
%	CONTENT
%-------------------------------------------------------------------------------
\begin{cventries}

%---------------------------------------------------------

  \cventry
  {PhD in Computer Science} % Degree
  {Reykjavík University} % Institution
  {Reykjavík, Iceland} % Location
  {Aug. 2014 - Present} % Date(s)
  {
    \begin{cvitems} % Description(s) bullet points
      \item {
      Supervisors: Prof H. Ulfarsson, Prof A. Claesson and Prof M. Albert
      }
      \item {
      Research Project: ``Finding structure in sets of permutations'' -
      \textit{The main goal is to develop an algorithm which will aid
      researchers in finding structures in sets of permutations and use those
      structures to find generating functions to enumerate the set. My
      assignment is primarily the development of the theory of permutation
      patterns relating to the new algorithm as well as implementation of the
      algorithm.}
      }
      \item {
      Selected courses: Algorithms, Data Structures, Theory of Computation,
      Datamining and Machine Learning
      }
    \end{cvitems}
  }

%---------------------------------------------------------

  \cventry
    {MMath (Hons) in Mathematics (1st Class)} % Degree
    {Universtiy of St Andrews} % Institution
    {Fife, Scotland} % Location
    {Sep. 2010 - Jun. 2014} % Date(s)
    {
      \begin{cvitems} % Description(s) bullet points
        \item {
        Supervisor: Dr M. Quick
        }
        \item {
        Dissertation: ``Powerful $p$-Groups'' - \textit{We will look into
        nilpotent groups and discover that all finite $p$-groups are nilpotent.
        We will look at powerful $p$-groups and explore their similarities with
        abelian $p$-groups. There will be a close look at the group
        $U_­­r(\mathbb{F}_p)$ in which we find its lower central series and show
        it is not powerful. We will show that all finite $p$-groups have a
        powerful normal subgroup whose index is bounded by a function of the
        group's rank and $p$.}
        }
        \item {
        Selected courses: Advanced Combinatorics, Semigroups, Topics in Groups,
        Finite Maths, Graph Theory, Symbolic Computation, Asymptotic Methods,
        Advanced Analytic Techniques, Numerical Analysis
        }
      \end{cvitems}
    }

%---------------------------------------------------------
\end{cventries}
